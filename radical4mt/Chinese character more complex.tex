\documentclass[11pt]{extarticle}

% packages to use Chinese characters in xelatex
% consider to use fonts with a good coverage of CJK
% Example are HanaMin or Hanamin AFDKO
% http://fonts.jp/hanazono/
% https://github.com/cjkvi/HanaMinAFDKO/releases
\usepackage{xeCJK}
\usepackage{newunicodechar}

% Main font for CJK
\setCJKmainfont{Hanazono Mincho A}

% This way you import new fonts (outside CJK range)
\newfontfamily\SourceHanSerif{Source Han Serif}

% This way you import new fonts for CJK
\newCJKfontfamily\HamazonoMinB{Hanazono Mincho B}
\newCJKfontfamily\HamazonoMinC{Hanazono Mincho C}
\newCJKfontfamily\IPAexMincho{IPAexMincho}



% If a character is missing in the main font you can patch it up using this command
% In this example 𠚍 and 𠂭 use the Hanazo Mincho B font
\newunicodechar{𠚍}{{\HamazonoMinB 𠚍}}
\newunicodechar{𠂭}{{\HamazonoMinB 𠂭}}
\newunicodechar{𬋢}{{\HamazonoMinC 𬋢}}
\newunicodechar{①}{{\SourceHanSerif ①}}
\newunicodechar{⑲}{{\SourceHanSerif ⑲}}
\newunicodechar{⑳}{{\SourceHanSerif ⑳}}

% To detect what font has a character one possiblity is:
% . open libreoffice writer
% . type the character
% . if the system shows it properly, export the file to pdf
% . use pdffonts to see what fonts are inside the pdf
% . the name will be mangled, but hopefully detectable in the fc-list output


\usepackage{url}
\begin{document}

\section{Example}

For the Chinese Characters decompose, we used the IDS data for CJK Unified Ideographs\footnote{\url{https://github.com/cjkvi/cjkvi-ids}}; in the repo the \texttt{ids.txt} file contains a mapping between almost 90 thousands Chinese Japanese or Korean characters and radicals and their decomposition.  For the characters with multiple decompositions we used the one from mainland China.  For example 乢 maps to ⿰山乚; complex characters can often decomposed further.  For example 鬱 maps to ⿳⿲木缶木冖⿰鬯彡, its component 鬯 maps to ⿱𠚍匕, and again 𠚍 maps to ⿶凵𠂭.

The circled integer (①--⑳) are used to denote the number strokes where the representing character is not known or not available to Unicode. For example 𬋢 expands to ⿱⿰禾⑲火.


\end{document}

